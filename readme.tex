\documentclass[fontsize=14pt]{scrartcl}
\setkomafont{disposition}{\normalfont\bfseries}

\usepackage{amsthm}
\usepackage{amsmath}
\usepackage{amssymb}
\usepackage{fancyhdr}
\usepackage{mathtools}
\usepackage{pdfrender}
\usepackage{mathrsfs}
\usepackage{array}

\usepackage[margin=1in]{geometry}

\theoremstyle{definition}
\newtheorem{problem-internal}{Problem}
\pagestyle{fancy}
\fancyhf{}
\renewcommand{\headrulewidth}{0pt}
\renewcommand*{\thefootnote}{\fnsymbol{footnote}}
\newcolumntype{L}{>{$}l<{$}}
\newcolumntype{C}{>{$}c<{$}}
\newcolumntype{R}{>{$}r<{$}}
\newcommand{\lcm}{\textup{lcm}}
\newcommand{\id}{\textup{id}}
\newcommand{\Eucl}{\textup{Eucl}}
\DeclarePairedDelimiter\ceil{\lceil}{\rceil}
\DeclarePairedDelimiter\floor{\lfloor}{\rfloor}



\newenvironment{problem}{
\medskip
\begin{problem-internal}
}{
\end{problem-internal}
}

\newenvironment{solution}{
\begin{proof}[Solution]
\vspace{-8px}
\setlength{\parskip}{4px}
\setlength{\parindent}{0px}
}{
\end{proof}
}


\begin{document}
\section*{}
\setcounter{section}{0}
\setcounter{problem-internal}{0}
\begin{problem}
Hello.\\\\
I will call the inputted function $\phi$. For readability, I will use the following notation to represent the modulo operator:
\begin{gather*}
[y]_x = y \mod x.
\end{gather*}
The program will then generate a linked list of the following form:
\begin{gather*}
[y_1]_x\rightarrow [y_2]_x\rightarrow [y_3]_x\rightarrow\hdots
\end{gather*}
where $y_n=\phi(y_{n-1})=\phi^{n-1}(y_1)$. This means that the generated linked list can be viewed as the following:
\begin{gather*}
[y_1]_x\rightarrow[\phi(y_1)]_x\rightarrow[\phi(\phi(y_1))]_x\rightarrow\hdots
\end{gather*}
It generates linked lists for all integers $x$ from 1 to 1000. 
Typically, $y_1=5$. The program will then find cycles in the linked list, i.e., find periodic points. I call $[y_n]_x$ a \textbf{periodic point} mod $x$ if there exists an integer $m$ such that $[y_n]_x=[\phi^m(y_n)]_x$, i.e., the value appears in the linked list again $m$ nodes later.\\
The program will then generate a plot of these periodic points. If $([y_n]_x, x)$ is a plot on the graph, that means that $[y_n]_x$ is a periodic point mod $x$. Notice that $[y_n]_x<x$, so naturally the graphs will have a lower triangular form.
If a linear function $\ell(x)$ is inputted ($\ell(x)=(-1)x+481$ is used in the below example) it will only plot periodic points $([y_n]_x,x)$ if $[y_n]_x=\ell(x)$. This was to visualize the lines that mysteriously appear in the graphs.\\
The goal of the project was not to find why the lines appeared, but to create a program which plots them quickly. It seems there are fractals and chaos involved ($y_1$ being the sensitive initial condition) due to how similar it seems to plotting periodic points on logistic bifurcation plots, but I am not sure.
\end{problem}

\end{document}
